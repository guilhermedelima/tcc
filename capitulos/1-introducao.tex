\chapter{Introdução}
 
A tecnologia nunca foi tão presente na sociedade como nos dias atuais. Não apenas pessoas são responsáveis por produzir informações, equipamentos eletrônicos também tornaram-se grandes criadores de dados, como por exemplo, registros de logs de servidores, sensores que são instalados nos mais variados contextos, entre uma infinidade aplicações. Mensurar o volume de todos estes registros eletrônicos não é uma tarefa fácil. \citeonline{white2012} apresenta alguns exemplos de como grandes empresas geram quantidades extremamente grandes de dados. O Facebook\footnote{Umas das redes sociais de maior sucesso no mundo.} chega a armazenar 10 bilhões de fotos, totalizando 1 petabyte, já a organização Internet Archive\footnote{Organização sem fins lucrativos responável por armazenar recursos multimídia.} contém cerca de 2 petabytes de dados, com uma taxa de crescimento de 20 terabytes por mês.

Segundo levantamento feito pela IDC~\cite{gantz2011}, a quantidade de informações capturadas, criadas, ou replicadas no universo digital ultrapassou a barreira de 1 zettabyte, equivalente a 1 trilhão de gigabytes. Entre o período de 2006 a 2011 a volume total de registros se multiplicava por nove a cada ano. O número de bits de todos estes dados pode ser comparado ao número de estrelas contidas no universo físico.

O grande problema consiste tornar processável toda esta gama de informações, que podem estar persistidas de forma estruturada, semi estruturada, ou sem nenhuma organização. De acordo com \citeonline{ibm2011}, o termo Big Data se aplica a todo este potencial de dados que não são passíveis de análise ou processamento através dos métodos e ferramentas tradicionais. Por muito tempo várias empresas tinham a liberdade de ignorar o uso de grande parte deste volume de informações, pois não havia como armazenar estes dados a um custo benefício aceitável. Todavia com o avanço das tecnologias relacionadas a Big Data percebeu-se que a análise da maioria destas informações pode trazer benefícios consideráveis, agregando novas percepções jamais imaginadas.

A análise realizada sobre petabytes de informações pode ser um fator determinante para a tomadas de decisões em vários contextos. A utilização desta tecnologia associada a algoritmos de aprendizagem de máquinas, como por exemplo, pode revelar tendências de usuários, necessidades em determinadas áreas que não se imaginaria sem uso desta abordagem. Com isso fica claro que o termo Big Data não está relacionado a apenas o armazenamento em grande escala, mas também ao processamento destes dados para agregar valor ao contexto em que for aplicado.

As tecnologias de Big Data descrevem uma nova geração de arquiteturas, projetadas para economicamente extrair valor de um grande volume, sobre uma grande variedade de dados, permitindo alta velocidade de captura, e/ou análise \cite{gantz2011}. Nesta definição é possível identificar os três pilares deste novo movimento: velocidade, volume e variedade. De acordo com \citeonline{ibm2011}, isto pode ser caracterizado com os três Vs presentes nas tecnologias Big Data.

O projeto Apache Hadoop\footnote{\url{http://hadoop.apache.org/}} é uma das soluções mais conhecidas para Big Data atualmente. Sua finalidade é oferecer uma infraestrutura para armazenamento e processamento de grande volume de dados, provendo escalabilidade linear e tolerância a falhas. Segundo \citeonline{white2012}, em abril de 2008 o Hadoop quebrou o recorde mundial e se tornou o sistema mais rápido a ordenar 1 terabyte, utilizando 910 máquinas essa marca foi atingida em 209 segundos. Em 2009 este valor foi reduzido para 62 segundos. Neste trabalho será apresentado um estudo sobre a arquitetura e o modelo proposto por este software.

%------------------------------------------------------------------------------%
 
\section{Objetivos}

\subsection{Objetivos Gerais}
 
Para este trabalho de conclusão de curso, o principal objetivo é compreender as tecnologias que envolvem o projeto Hadoop, uma das principais soluções existentes para análise de dados dentro do contexto de Big Data.
 
 
%------------------------------------------------------------------------------%
 
\subsection{Específicos}
 
Os objetivos específicos desse trabalho são apresentados a seguir:
 
\begin{enumerate}
  \item Analisar a arquitetura e funcionamento do núcleo do Hadoop.
  \item Descrever os passos necessários para o desenvolvimento de aplicações utilizando o paradigma MapReduce.
  \item Analisar as ferramentas que fazem parte do ecossistema Hadoop e oferecem serviços executados no topo desta arquitetura.
  \item Construção de senso crítico para análise dos cenários Big Data e as soluções possíveis para estes diferentes contextos.
\end{enumerate}
 
%------------------------------------------------------------------------------%
 
\section{Organização do Trabalho}
 
O primeiro passo para a construção deste trabalho ocorreu através de uma revisão bibliográfica sobre as tecnologias que se encaixam nesta nova abordagem Big Data. Com as pesquisas realizadas o Hadoop se apresentou como uma das ferramentas mais utilizadas e consolidadas para este contexto. Portanto a próxima etapa consistiu em uma pesquisa para compreender e detalhar sua arquitetura e funcionamento. Em seguida foi realizada uma descrição dos passos necessários para o desenvolvimento para aplicações MapReduce e também dos projetos que compõe o ecossistema Hadoop.

Os capítulos subsequentes estão estruturados de acordo com a seguinte organização. O capítulo 2 apresenta uma descrição detalhada sobre o Hadoop, onde o foco está no sistema de arquivos distribuídos HDFS e também no \textit{framework} para computação paralela MapReduce. 

No capítulo 3 são discutidos os passos necessários para o desenvolvimento de software utilizando o \textit{framework} MapReduce, onde são abordados aspectos técnicos e práticos que envolvem esta abordagem.

O capítulo 4 apresenta os projetos que fazem parte do ecossistema Hadoop. O foco desta sessão está nas ferramentes construídas no topo da arquitetura Hadoop, em específico o banco de dados NoSQL HBase e também o \textit{data warehouse} distribuído Hive.

No capítulo 5 são discutidos os resultados obtidos após o término deste trabalho, onde são apresentados os próximos passos para a conclusão do TCC 2.


