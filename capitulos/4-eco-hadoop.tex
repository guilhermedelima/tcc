\chapter{Ecossistema Hadoop}
\label{cap:eco}

Os capítulos anteriores abordaram os principais componentes do Hadoop: o sistema de arquivos distribuídos HDFS e o framework MapReduce. Estes podem ser considerados o núcleo de todo o sistema, porém o software Hadoop também é composto por um conglomerado de projetos que fornecem serviços relacionados a computação distribuída em larga escala, formando o ecossistema Hadoop. A tabela \ref{tab-eco} apresenta alguns projetos que estão envolvidos neste contexto.

\begin{savenotes}
\begin{table}[!ht]
\begin{center}
  \begin{tabular}{|p{3cm}|p{7cm}|}
	\hline
	Projeto & Descrição
	\\ \hline
	Common & Conjunto de componentes e interfaces para sistemas de arquivos distribuídos e operações de Entrada/Saída.
	\\ \hline
	Avro\footnote{\url{http://avro.apache.org/}} & Sistema para serialização de dados.
	\\ \hline
	HDFS & Sistema de arquivos distribuídos executado sobre clusters com máquinas de baixo custo
	\\ \hline
	MapReduce & Framework para processamento distribuído de dados, aplicado em clusters com máquinas de baixo custo.
	\\ \hline
	Pig\footnote{\url{http://pig.apache.org/}} & Linguagem de procedimentos de alto nível para grandes bases de dados. Executada em clusters HDFS e MapReduce.
	\\ \hline
	Hive\footnote{\url{http://hive.apache.org/}} & Um data warehouse distribuído. Gerencia arquivos no HDFS e provê linguagem de consulta baseada em SQL.
	\\ \hline
	HBase\footnote{\url{http://hbase.apache.org/}} & Banco de dados distribuído orientado a colunas. Utiliza o HDFS para armazenamento dos dados.
	\\ \hline
	ZooKeeper\footnote{\url{http://zookeeper.apache.org/}} & Coordenador de serviços distribuídos.
	\\ \hline
	Sqoop\footnote{\url{http://sqoop.apache.org/}} & Ferramenta para mover dados entre banco relacionais e o HDFS.
	\\ \hline
  \end{tabular}
  \caption{Ecossistema Hadoop, retirado de 
  \citeonline{white2012}, \citeonline{shvachko2010}}
\label{tab-eco}
\end{center}
\end{table}
\end{savenotes}

Apesar do Hadoop apresentar uma boa alternativa para processamento em larga escala, ainda existem algumas limitações em seu uso. Como discutido anteriormente o HDFS foi projetado de acordo com o padrão \textit{write-once}, \textit{read-many-times}, desta forma não há acesso randômico para operações de leitura e escrita. Outro aspecto negativo se dá pelo baixo nível requerido para o desenvolvimento neste tipo de ambiente. Segundo \citeonline{thusoo2009}, a utilização do \textit{framework} MapReduce faz com que programadores implementem aplicações difíceis de realizar manutenção e reuso de código,

Alguns dos projetos do ecossistema citado na tabela \ref{tab-eco} foram criados justamente para resolver estes problemas, desta forma utilizam-se do Hadoop para prover serviços com um nível de abstração maior para o usuário. Neste capítulo discutimos sobre o banco de dados orientado a colunas HBase e também sobre a ferramenta de data warehouse distribuído Hive.

\section{HBase}

Os bancos de dados relacionais sempre desempenharam um papel importante no design e implementação dos negócios da maioria das empresas. As necessidades para este contexto sempre envolveram o registro de informações de usuários, produtos, entre inúmeros exemplos. Este tipo de arquitetura oferecida pelos SGBDs foram construídas de acordo com o modelo de transações definido pelas propriedades ACID. Segundo \citeonline{george2011}, desta forma é possível garantir que os dados sejam fortemente consistentes, o que parece ser um requisito bastante favorável. Esta abordagem funciona bem enquanto os dados armazenados são relativamente pequenos, porém o crescimento desta demanda pode ocasionar sérios problemas estruturais.

De acordo com \citeonline{george2011}, os bancos de dados relacionais não estão preparados para análise de grande volume de dados caracterizados pelo contexto Big Data. É possível encontrar soluções que se adaptem a esta necessidade, porém na maioria das vezes envolvem mudanças drásticas e complexas na arquitetura e também possuem um custo muito alto, já que em muitos casos a resposta está ligada diretamente ao uso de escalabilidade vertical, ou seja, ocorre com a compra de máquinas caras e computacionalmente poderosas. Contudo não há garantias de que com um aumento ainda maior da quantidade de dados todos os problemas inicias não voltem a acontecer, isso porque a relação entre o uso de transações e o volume de informações processadas não é linear.

\subsection{NoSQL}

Um novo movimento denominado NoSQL surgiu com propósito de solucionar os problemas  descritos na sessão anterior. Segundo \citeonline{cattell2011}, não há um consenso sobre o significado deste termo, esta nomenclatura pode ser interpretada como not only SQL (do inglês, não apenas SQL), ou também como uma forma de explicitar o não uso da abordagem relacionada aos bancos de dados relacionais. De acordo com \citeonline{george2011}, as tecnologias NoSQL devem ser compreendidas como um complemento ao uso dos SGBDs tradicionais, ou seja, esta abordagem não é revolucionária e sim evolucionária. Uma das principais características destes sistemas é descrita por \citeonline{cattell2011} como a habilidade em prover escalabilidade horizontal para operações e armazenamento de dados ao longo de vários servidores, ou seja, permitir de maneira eficiente a inclusão de novas máquinas no sistema para melhora de performance, ao invés do uso da escalabilidade vertical, na qual procura-se aumentar a capacidade de processamento através do compartilhamento de memória RAM entre os computadores ou com a compra de máquinas de alto custo financeiro.

Ao contrário dos bancos de dados relacionais, os novos sistemas NoSQL diferem entre si quanto aos tipos de dados que são suportados, não havendo uma terminologia padrão. A pesquisa realizada por \citeonline{cattell2011} classifica estas novas tecnologias de acordo o modelo de dados utilizados por cada uma delas. As principais categorias são definidas a seguir:

\begin{itemize}

  \item{Armazenamento chave/valor: este tipo de sistema NoSQL armazena valores e um índice para encontrá-los, no qual é baseado em uma chave definida pelo programador. Exemplos: Voldemort\footnote{\url{http://www.project-voldemort.com/voldemort/}}, Riak\footnote{\url{http://basho.com/riak/}};}
  \item{Registro de Documentos: sistemas que armazenam documentos indexados, provendo uma linguagem simples para consulta. Documentos, ao contrário de tuplas, não são definidos por um esquema fixo, podendo ser associados a diferentes valores. Exemplos: CouchDB\footnote{\url{http://couchdb.apache.org/}}, MongoDB\footnote{\url{http://www.mongodb.org/}};}
  \item{Armazenamento orientado a colunas: sistemas que armazenam os dados em formato de tabelas, porém as colunas são agrupadas em famílias e espalhadas horizontalmente ao longo das máquinas da rede. Exemplos: Google Bigtable, HBase;}
  \item{Bancos de dados de grafos: sistemas que permitem uma eficiente distribuição e consulta dos dados armazenados em forma de grafos. Exemplo: Neo4J\footnote{\url{http://www.neo4j.org/}};}

\end{itemize}








