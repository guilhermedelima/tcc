\begin{resumo}[Abstract]
 \begin{otherlanguage*}{english}

The storage of information in digital format has grown considerably in recent years. Not only people are responsible for producing data, electronic equipment have also become major generators of records, such as servers, GPS devices, computers scattered in various contexts, among a multitude of applications. The term Big Data refers to all this amount of data that is on the order of petabytes and can not be analyzed by traditional methods. In this paper, we present a study of one of the most known architectures to solve these problems, Hadoop software. The development MapReduce paradigm is discussed, as well as designs that are built on top of Hadoop system providing service at a higher abstraction level. Beyond the research stage, an Big Data application architecture facing a real case study is defined and implemented, which involves the extraction and analysis of social networking publications focusing on Brazilian politics.

   \vspace{\onelineskip}
 
   \noindent 
   \textbf{Key-words}: Big Data. Hadoop. Distributed Computing.
 \end{otherlanguage*}
\end{resumo}
